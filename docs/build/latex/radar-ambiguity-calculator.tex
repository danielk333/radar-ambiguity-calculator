%% Generated by Sphinx.
\def\sphinxdocclass{report}
\documentclass[letterpaper,10pt,english]{sphinxmanual}
\ifdefined\pdfpxdimen
   \let\sphinxpxdimen\pdfpxdimen\else\newdimen\sphinxpxdimen
\fi \sphinxpxdimen=.75bp\relax

\PassOptionsToPackage{warn}{textcomp}
\usepackage[utf8]{inputenc}
\ifdefined\DeclareUnicodeCharacter
 \ifdefined\DeclareUnicodeCharacterAsOptional
  \DeclareUnicodeCharacter{"00A0}{\nobreakspace}
  \DeclareUnicodeCharacter{"2500}{\sphinxunichar{2500}}
  \DeclareUnicodeCharacter{"2502}{\sphinxunichar{2502}}
  \DeclareUnicodeCharacter{"2514}{\sphinxunichar{2514}}
  \DeclareUnicodeCharacter{"251C}{\sphinxunichar{251C}}
  \DeclareUnicodeCharacter{"2572}{\textbackslash}
 \else
  \DeclareUnicodeCharacter{00A0}{\nobreakspace}
  \DeclareUnicodeCharacter{2500}{\sphinxunichar{2500}}
  \DeclareUnicodeCharacter{2502}{\sphinxunichar{2502}}
  \DeclareUnicodeCharacter{2514}{\sphinxunichar{2514}}
  \DeclareUnicodeCharacter{251C}{\sphinxunichar{251C}}
  \DeclareUnicodeCharacter{2572}{\textbackslash}
 \fi
\fi
\usepackage{cmap}
\usepackage[T1]{fontenc}
\usepackage{amsmath,amssymb,amstext}
\usepackage{babel}
\usepackage{times}
\usepackage[Bjarne]{fncychap}
\usepackage{sphinx}

\usepackage{geometry}

% Include hyperref last.
\usepackage{hyperref}
% Fix anchor placement for figures with captions.
\usepackage{hypcap}% it must be loaded after hyperref.
% Set up styles of URL: it should be placed after hyperref.
\urlstyle{same}
\addto\captionsenglish{\renewcommand{\contentsname}{Contents:}}

\addto\captionsenglish{\renewcommand{\figurename}{Fig.}}
\addto\captionsenglish{\renewcommand{\tablename}{Table}}
\addto\captionsenglish{\renewcommand{\literalblockname}{Listing}}

\addto\captionsenglish{\renewcommand{\literalblockcontinuedname}{continued from previous page}}
\addto\captionsenglish{\renewcommand{\literalblockcontinuesname}{continues on next page}}

\addto\extrasenglish{\def\pageautorefname{page}}

\setcounter{tocdepth}{1}



\title{radar-ambiguity-calculator Documentation}
\date{Dec 10, 2018}
\release{0.1}
\author{Daniel Kastinen \& Felipe Betancourt}
\newcommand{\sphinxlogo}{\vbox{}}
\renewcommand{\releasename}{Release}
\makeindex

\begin{document}

\maketitle
\sphinxtableofcontents
\phantomsection\label{\detokenize{index::doc}}



\chapter{Introduction}
\label{\detokenize{introduction:introduction}}\label{\detokenize{introduction::doc}}
An ambiguity problem arises then determining the position and motion of objects with a radar system. The ambiguity problem
is translated in the fact that different Direction of Arrival (DOA) can lead to the same response. In  the paper
\sphinxstyleemphasis{Determining all ambiguities in direction of arrival measured by radar systems} by Daniel Kastinen, a mathematical
framework and practical method to find all ambiguities in any multichannel system is described. The new formulation
allows for an efficient implementation using the numerical Moore-Penrose inverse to find all
ambiguities and approximate ambiguities.

Here two main functions composed by other several functions are developed to implement the algorithm proposed and
retrieve solutions. The \sphinxstyleemphasis{ambiguity\_calculator.py} file contains a function that by entering the name of a listed radar
configuration, described in \sphinxstyleemphasis{radarconf.py} and operating frequency is able to calculate all ambiguities of a radar
system and saves the results into an HDF5 file for later processing. \sphinxstyleemphasis{plots\_generator.py} contains a function that
generates the solution and plots for a certain radar configuration, with a certain operating frequency and for a given
signal wave (described as a wave vector). All of the use user defined functions in \sphinxstyleemphasis{functions.py}.


\chapter{Tool for radar ambiguity solution}
\label{\detokenize{radar:module-ambiguity_calculator}}\label{\detokenize{radar:tool-for-radar-ambiguity-solution}}\label{\detokenize{radar::doc}}\index{ambiguity\_calculator (module)}\index{ambiguities\_calculate() (in module ambiguity\_calculator)}

\begin{fulllineitems}
\phantomsection\label{\detokenize{radar:ambiguity_calculator.ambiguities_calculate}}\pysiglinewithargsret{\sphinxcode{\sphinxupquote{ambiguity\_calculator.}}\sphinxbfcode{\sphinxupquote{ambiguities\_calculate}}}{\emph{radar\_name}, \emph{frequency}}{}
Calculates the solution for the radar ambiguity problem by implementing developed by Daniel Kastinen in his paper
“Determining all ambiguities in direction of arrival measured by radar systems”. The output of the calculations is
summarized in a .h5 file with HDF5 format and saved into the /processed\_data folder
\begin{quote}\begin{description}
\item[{Parameters}] \leavevmode\begin{itemize}
\item {} 
\sphinxstyleliteralstrong{\sphinxupquote{radar\_name}} \textendash{} Name of the radar to be studied out of a list of defined configurations.

\item {} 
\sphinxstyleliteralstrong{\sphinxupquote{frequency}} \textendash{} Operating frequency {[}MHz{]}

\end{itemize}

\end{description}\end{quote}

\end{fulllineitems}



\chapter{Tool for generating plots}
\label{\detokenize{plots_generator:module-plots_generator}}\label{\detokenize{plots_generator:tool-for-generating-plots}}\label{\detokenize{plots_generator::doc}}\index{plots\_generator (module)}\index{generate\_plots() (in module plots\_generator)}

\begin{fulllineitems}
\phantomsection\label{\detokenize{plots_generator:plots_generator.generate_plots}}\pysiglinewithargsret{\sphinxcode{\sphinxupquote{plots\_generator.}}\sphinxbfcode{\sphinxupquote{generate\_plots}}}{\emph{radar\_name}, \emph{frequency}, \emph{elevation}, \emph{azimuth}}{}
Import results summary from ambiguity\_calculator algorithm and plot results for a DOA given by azimuth and elevation
angles.
\begin{quote}\begin{description}
\item[{Parameters}] \leavevmode\begin{itemize}
\item {} 
\sphinxstyleliteralstrong{\sphinxupquote{radar\_name}} \textendash{} choose radar configuration

\item {} 
\sphinxstyleliteralstrong{\sphinxupquote{elevation}} \textendash{} DOA elevation angle {[}º{]}

\item {} 
\sphinxstyleliteralstrong{\sphinxupquote{azimuth}} \textendash{} DOA azimuth angle {[}º{]}

\end{itemize}

\end{description}\end{quote}

\end{fulllineitems}



\chapter{List of functions used during the calculations}
\label{\detokenize{functions:module-functions}}\label{\detokenize{functions:list-of-functions-used-during-the-calculations}}\label{\detokenize{functions::doc}}\index{functions (module)}\index{R\_cal() (in module functions)}

\begin{fulllineitems}
\phantomsection\label{\detokenize{functions:functions.R_cal}}\pysiglinewithargsret{\sphinxcode{\sphinxupquote{functions.}}\sphinxbfcode{\sphinxupquote{R\_cal}}}{\emph{sensor\_groups}, \emph{xycoords}}{}~\begin{quote}\begin{description}
\item[{Parameters}] \leavevmode\begin{itemize}
\item {} 
\sphinxstyleliteralstrong{\sphinxupquote{sensor\_groups}} \textendash{} how many sensor groups are there in the radar configuration. Do not count on the one located     at the origin

\item {} 
\sphinxstyleliteralstrong{\sphinxupquote{xycoords}} \textendash{} locations of the subgroups {[}m{]}

\end{itemize}

\item[{Return R}] \leavevmode
subgroup phase center

\end{description}\end{quote}

\end{fulllineitems}

\index{explicit() (in module functions)}

\begin{fulllineitems}
\phantomsection\label{\detokenize{functions:functions.explicit}}\pysiglinewithargsret{\sphinxcode{\sphinxupquote{functions.}}\sphinxbfcode{\sphinxupquote{explicit}}}{\emph{intersection\_line}, \emph{intersections\_ind}, \emph{cap\_intersections\_of\_slines}, \emph{xy}, \emph{k0}}{}
Calculation of ambiguities
\begin{quote}\begin{description}
\item[{Parameters}] \leavevmode\begin{itemize}
\item {} 
\sphinxstyleliteralstrong{\sphinxupquote{intersection\_line}} \textendash{} matrix of intersection lines

\item {} 
\sphinxstyleliteralstrong{\sphinxupquote{intersections\_ind}} \textendash{} valid intersection indexes

\item {} 
\sphinxstyleliteralstrong{\sphinxupquote{cap\_intersections\_of\_slines}} \textendash{} cap?

\item {} 
\sphinxstyleliteralstrong{\sphinxupquote{xy}} \textendash{} xy coordinates of radar subgroups

\item {} 
\sphinxstyleliteralstrong{\sphinxupquote{k0}} \textendash{} signal wave vector

\end{itemize}

\item[{Return ambiguity\_distances\_explicit}] \leavevmode
ambiguity distances

\item[{Return ambiguity\_normal\_explicit}] \leavevmode
\item[{Return k\_finds}] \leavevmode
\end{description}\end{quote}

\end{fulllineitems}

\index{intersections\_cal() (in module functions)}

\begin{fulllineitems}
\phantomsection\label{\detokenize{functions:functions.intersections_cal}}\pysiglinewithargsret{\sphinxcode{\sphinxupquote{functions.}}\sphinxbfcode{\sphinxupquote{intersections\_cal}}}{\emph{pinv\_norm}, \emph{PERMS\_J}, \emph{intersection\_line}, \emph{R}, \emph{**kwargs}}{}
Given that the Moore-Penrose solution gives a solution for any case. It is necessary to choose the ones whose     error is below a given tolerance value.
\begin{quote}\begin{description}
\item[{Parameters}] \leavevmode\begin{itemize}
\item {} 
\sphinxstyleliteralstrong{\sphinxupquote{pinv\_norm}} \textendash{} distance from real b\_vector and the one obtained by the moore\_penrose solution

\item {} 
\sphinxstyleliteralstrong{\sphinxupquote{PERMS\_J}} \textendash{} Valid combinations

\item {} 
\sphinxstyleliteralstrong{\sphinxupquote{intersection\_line}} \textendash{} matrix of intersection lines

\item {} 
\sphinxstyleliteralstrong{\sphinxupquote{R}} \textendash{} subgroup phase center

\item {} 
\sphinxstyleliteralstrong{\sphinxupquote{kwargs}} \textendash{} if ‘norm’ an extra condition will be applied and is that if pinv\_norm is greater than zero, it     will disregarded.

\end{itemize}

\item[{Return intersections}] \leavevmode
dictionary with the so far valid intersection combinations.

\end{description}\end{quote}

\end{fulllineitems}

\index{k0\_cal() (in module functions)}

\begin{fulllineitems}
\phantomsection\label{\detokenize{functions:functions.k0_cal}}\pysiglinewithargsret{\sphinxcode{\sphinxupquote{functions.}}\sphinxbfcode{\sphinxupquote{k0\_cal}}}{\emph{el0}, \emph{az0}}{}
Calculation of wave vector defined defined in paper.
\begin{quote}\begin{description}
\item[{Parameters}] \leavevmode\begin{itemize}
\item {} 
\sphinxstyleliteralstrong{\sphinxupquote{el0}} \textendash{} elevation angle {[}º{]}

\item {} 
\sphinxstyleliteralstrong{\sphinxupquote{az0}} \textendash{} azimuth angle {[}º{]}

\end{itemize}

\end{description}\end{quote}

\end{fulllineitems}

\index{lambda\_cal() (in module functions)}

\begin{fulllineitems}
\phantomsection\label{\detokenize{functions:functions.lambda_cal}}\pysiglinewithargsret{\sphinxcode{\sphinxupquote{functions.}}\sphinxbfcode{\sphinxupquote{lambda\_cal}}}{\emph{frequency}}{}
Calculate the wave length of electromagnetic radiation given its frequency.
\begin{quote}\begin{description}
\item[{Parameters}] \leavevmode
\sphinxstyleliteralstrong{\sphinxupquote{frequency}} \textendash{} Frequency of electromagnetic radiation {[}MHz{]}

\item[{Return wavelength}] \leavevmode
wave length of electromagnetic radiation {[}m{]}

\end{description}\end{quote}

\end{fulllineitems}

\index{linCoeff\_cal() (in module functions)}

\begin{fulllineitems}
\phantomsection\label{\detokenize{functions:functions.linCoeff_cal}}\pysiglinewithargsret{\sphinxcode{\sphinxupquote{functions.}}\sphinxbfcode{\sphinxupquote{linCoeff\_cal}}}{\emph{R}}{}
Calculate linear coefficients given R.
\begin{quote}\begin{description}
\item[{Parameters}] \leavevmode
\sphinxstyleliteralstrong{\sphinxupquote{R}} \textendash{} matrix of subgroup phase centers

\end{description}\end{quote}

\end{fulllineitems}

\index{mooore\_penrose\_solution() (in module functions)}

\begin{fulllineitems}
\phantomsection\label{\detokenize{functions:functions.mooore_penrose_solution}}\pysiglinewithargsret{\sphinxcode{\sphinxupquote{functions.}}\sphinxbfcode{\sphinxupquote{mooore\_penrose\_solution}}}{\emph{W}, \emph{b}}{}
Calculate the difference that produces the use of the Moore-Penrose solution matrix to the algebraic equation
\begin{quote}\begin{description}
\item[{Parameters}] \leavevmode\begin{itemize}
\item {} 
\sphinxstyleliteralstrong{\sphinxupquote{W}} \textendash{} W matrix as described in paper

\item {} 
\sphinxstyleliteralstrong{\sphinxupquote{b}} \textendash{} b vector as described in paper

\end{itemize}

\item[{Return intersection\_line}] \leavevmode
matrix of intersection lines

\item[{Return pinv\_norm}] \leavevmode
difference after solution check

\end{description}\end{quote}

\end{fulllineitems}

\index{mooore\_penrose\_solution\_par() (in module functions)}

\begin{fulllineitems}
\phantomsection\label{\detokenize{functions:functions.mooore_penrose_solution_par}}\pysiglinewithargsret{\sphinxcode{\sphinxupquote{functions.}}\sphinxbfcode{\sphinxupquote{mooore\_penrose\_solution\_par}}}{\emph{W}, \emph{b\_set}, \emph{pnum}, \emph{niter}, \emph{intersection\_line\_set}, \emph{pinv\_norm\_set}}{}
Calculate the Moore-Penrose solution for each case making use of parallel threading to parallelize task
and joining results. Calls the function moore\_penrose\_solution\_ptr
\begin{quote}\begin{description}
\item[{Parameters}] \leavevmode\begin{itemize}
\item {} 
\sphinxstyleliteralstrong{\sphinxupquote{W}} \textendash{} Matrix representation of all the normal vector of the planes going through the intersection.

\item {} 
\sphinxstyleliteralstrong{\sphinxupquote{b\_set}} \textendash{} Set ob b vectors for which the solution is going to be computed.

\item {} 
\sphinxstyleliteralstrong{\sphinxupquote{pnum}} \textendash{} Number of cores for parallelizing

\item {} 
\sphinxstyleliteralstrong{\sphinxupquote{niter}} \textendash{} number of permutations.

\item {} 
\sphinxstyleliteralstrong{\sphinxupquote{intersection\_line\_set}} \textendash{} initial array with zeros where the intersection lines will be allocated.

\item {} 
\sphinxstyleliteralstrong{\sphinxupquote{pinv\_norm\_set}} \textendash{} error in the MP solution approximation

\end{itemize}

\end{description}\end{quote}

\end{fulllineitems}

\index{mooore\_penrose\_solution\_ptr() (in module functions)}

\begin{fulllineitems}
\phantomsection\label{\detokenize{functions:functions.mooore_penrose_solution_ptr}}\pysiglinewithargsret{\sphinxcode{\sphinxupquote{functions.}}\sphinxbfcode{\sphinxupquote{mooore\_penrose\_solution\_ptr}}}{\emph{W}, \emph{Wpinv}, \emph{b\_set}, \emph{intersection\_line\_set}, \emph{pinv\_norm\_set}, \emph{ind\_range}}{}
Calculate the difference tha produces the use of the Moore-Penrose solution matrix to the algebraic equation in     paper.
\begin{quote}\begin{description}
\item[{Parameters}] \leavevmode\begin{itemize}
\item {} 
\sphinxstyleliteralstrong{\sphinxupquote{W}} \textendash{} W matrix

\item {} 
\sphinxstyleliteralstrong{\sphinxupquote{Wpinv}} \textendash{} Pseudo-inverser of W matrix

\item {} 
\sphinxstyleliteralstrong{\sphinxupquote{b\_set}} \textendash{} set of b vctors

\item {} 
\sphinxstyleliteralstrong{\sphinxupquote{intersection\_line\_set}} \textendash{} matrix of intersection lines

\item {} 
\sphinxstyleliteralstrong{\sphinxupquote{pinv\_norm\_set}} \textendash{} matrix of pinv\_norm vectors

\item {} 
\sphinxstyleliteralstrong{\sphinxupquote{ind\_range}} \textendash{} index

\end{itemize}

\end{description}\end{quote}

\end{fulllineitems}

\index{nvec\_j() (in module functions)}

\begin{fulllineitems}
\phantomsection\label{\detokenize{functions:functions.nvec_j}}\pysiglinewithargsret{\sphinxcode{\sphinxupquote{functions.}}\sphinxbfcode{\sphinxupquote{nvec\_j}}}{\emph{j}, \emph{R}}{}
Normalized vector normal to plane j. Each plane is given as a column in R.
\begin{quote}\begin{description}
\item[{Parameters}] \leavevmode\begin{itemize}
\item {} 
\sphinxstyleliteralstrong{\sphinxupquote{j}} \textendash{} index

\item {} 
\sphinxstyleliteralstrong{\sphinxupquote{R}} \textendash{} subgroup phase center

\end{itemize}

\end{description}\end{quote}

\end{fulllineitems}

\index{p0\_jk() (in module functions)}

\begin{fulllineitems}
\phantomsection\label{\detokenize{functions:functions.p0_jk}}\pysiglinewithargsret{\sphinxcode{\sphinxupquote{functions.}}\sphinxbfcode{\sphinxupquote{p0\_jk}}}{\emph{j}, \emph{k}, \emph{R}, \emph{n0}, \emph{K}}{}
Displacement point
\begin{quote}\begin{description}
\item[{Parameters}] \leavevmode\begin{itemize}
\item {} 
\sphinxstyleliteralstrong{\sphinxupquote{j}} \textendash{} 

\item {} 
\sphinxstyleliteralstrong{\sphinxupquote{k}} \textendash{} 

\item {} 
\sphinxstyleliteralstrong{\sphinxupquote{R}} \textendash{} 

\item {} 
\sphinxstyleliteralstrong{\sphinxupquote{n0}} \textendash{} 

\item {} 
\sphinxstyleliteralstrong{\sphinxupquote{K}} \textendash{} 

\end{itemize}

\end{description}\end{quote}

\end{fulllineitems}

\index{permutations\_create() (in module functions)}

\begin{fulllineitems}
\phantomsection\label{\detokenize{functions:functions.permutations_create}}\pysiglinewithargsret{\sphinxcode{\sphinxupquote{functions.}}\sphinxbfcode{\sphinxupquote{permutations\_create}}}{\emph{permutations\_base}, \emph{intersections\_ind}, \emph{k\_length}, \emph{permutation\_index}}{}
Create all possible permutations by combining current permutation combinations with valid indexes and new
\begin{quote}\begin{description}
\item[{Parameters}] \leavevmode\begin{itemize}
\item {} 
\sphinxstyleliteralstrong{\sphinxupquote{permutations\_base}} \textendash{} set of combinations created so far

\item {} 
\sphinxstyleliteralstrong{\sphinxupquote{intersections\_ind}} \textendash{} indexes of valid combinations

\item {} 
\sphinxstyleliteralstrong{\sphinxupquote{k\_length}} \textendash{} set of elements to create new permutations, k

\item {} 
\sphinxstyleliteralstrong{\sphinxupquote{permutation\_index}} \textendash{} index of current permutation. Chooses k\_j

\end{itemize}

\end{description}\end{quote}

\end{fulllineitems}

\index{slines\_intersections() (in module functions)}

\begin{fulllineitems}
\phantomsection\label{\detokenize{functions:functions.slines_intersections}}\pysiglinewithargsret{\sphinxcode{\sphinxupquote{functions.}}\sphinxbfcode{\sphinxupquote{slines\_intersections}}}{\emph{k0}, \emph{intersections\_ind}, \emph{intersection\_line}, \emph{cutoff\_ph\_ang}}{}
Find all s-lines that intersect with the cap by range check.
\begin{quote}\begin{description}
\item[{Parameters}] \leavevmode\begin{itemize}
\item {} 
\sphinxstyleliteralstrong{\sphinxupquote{k0}} \textendash{} wave vector

\item {} 
\sphinxstyleliteralstrong{\sphinxupquote{intersections\_ind}} \textendash{} indexes of valid combinations

\item {} 
\sphinxstyleliteralstrong{\sphinxupquote{intersection\_line}} \textendash{} intersection lines matrix

\item {} 
\sphinxstyleliteralstrong{\sphinxupquote{cutoff\_ph\_ang}} \textendash{} cut-off angle

\end{itemize}

\end{description}\end{quote}

\end{fulllineitems}



\chapter{Tested radar configurations}
\label{\detokenize{radarconf:tested-radar-configurations}}\label{\detokenize{radarconf::doc}}
The user can test the algorithms with these radar configurations.

\phantomsection\label{\detokenize{radarconf:module-radarconf}}\index{radarconf (module)}\index{radar\_conf() (in module radarconf)}

\begin{fulllineitems}
\phantomsection\label{\detokenize{radarconf:radarconf.radar_conf}}\pysiglinewithargsret{\sphinxcode{\sphinxupquote{radarconf.}}\sphinxbfcode{\sphinxupquote{radar\_conf}}}{\emph{radar\_name}, \emph{frequency}}{}
Select radar radar configuration after name.
\begin{quote}\begin{description}
\item[{Parameters}] \leavevmode\begin{itemize}
\item {} 
\sphinxstyleliteralstrong{\sphinxupquote{radar\_name}} \textendash{} input by user

\item {} 
\sphinxstyleliteralstrong{\sphinxupquote{frequency}} \textendash{} frequency at which the radar array is being operated {[}Mhz{]}

\end{itemize}

\item[{Return lambda0}] \leavevmode
wave lenght corresponding to radar frequency {[}m{]}

\item[{Return xycoords}] \leavevmode
coordinates of subarray centers w.r.t. center of radar configuration in wavelengths.

\end{description}\end{quote}

\end{fulllineitems}



\chapter{Example of utilization and expected outputs.}
\label{\detokenize{test_example:example-of-utilization-and-expected-outputs}}\label{\detokenize{test_example::doc}}

\section{Solve the problem for a certain radar configuration}
\label{\detokenize{test_example:solve-the-problem-for-a-certain-radar-configuration}}
As an example take one of the radar configurations, \sphinxstylestrong{JONES} in this case, with a frequency of 31 MHz.

The coordinates of the subarray are


\begin{savenotes}\sphinxattablestart
\centering
\begin{tabulary}{\linewidth}[t]{|T|T|}
\hline
\sphinxstyletheadfamily 
x
&\sphinxstyletheadfamily 
y
\\
\hline
0
&
2
\\
\hline
0
&
-2.5
\\
\hline
-2
&
0
\\
\hline
2.5
&
0
\\
\hline
0
&
0
\\
\hline
\end{tabulary}
\par
\sphinxattableend\end{savenotes}

By running a python script with

\fvset{hllines={, ,}}%
\begin{sphinxVerbatim}[commandchars=\\\{\}]
\PYG{k+kn}{import} \PYG{n+nn}{os}
\PYG{k+kn}{import} \PYG{n+nn}{numpy} \PYG{k}{as} \PYG{n+nn}{np}
\PYG{k+kn}{from} \PYG{n+nn}{functions} \PYG{k}{import} \PYG{o}{*}
\PYG{k+kn}{from} \PYG{n+nn}{radarconf} \PYG{k}{import} \PYG{n}{radar\PYGZus{}conf}
\PYG{k+kn}{from} \PYG{n+nn}{ambiguity\PYGZus{}calculator} \PYG{k}{import} \PYG{n}{ambiguities\PYGZus{}calculate}
\PYG{k+kn}{import} \PYG{n+nn}{itertools}
\PYG{k+kn}{from} \PYG{n+nn}{scipy}\PYG{n+nn}{.}\PYG{n+nn}{constants} \PYG{k}{import} \PYG{n}{pi} \PYG{k}{as} \PYG{n}{pi}
\PYG{k+kn}{from} \PYG{n+nn}{time} \PYG{k}{import} \PYG{n}{time}\PYG{p}{,} \PYG{n}{gmtime}\PYG{p}{,} \PYG{n}{strftime}
\PYG{k+kn}{import} \PYG{n+nn}{h5py}

\PYG{n}{ambiguities\PYGZus{}calculate}\PYG{p}{(}\PYG{n}{radar\PYGZus{}name}\PYG{o}{=}\PYG{l+s+s1}{\PYGZsq{}}\PYG{l+s+s1}{JONES}\PYG{l+s+s1}{\PYGZsq{}}\PYG{p}{,} \PYG{n}{frequency}\PYG{o}{=}\PYG{l+m+mi}{31}\PYG{p}{)}
\end{sphinxVerbatim}

a HDF5 called JONES.h5 containing the calculation results is generated in the folder ../processed\_data/JONES.

JONES.h5 contains several items, data sets. Organized between two main HDF5 groups.
\begin{itemize}
\item {} 
root:
\begin{itemize}
\item {} 
trivial\_calculations: holds results from calculations which are straight forward and whose results can be used for to track the results.
\begin{itemize}
\item {} 
\sphinxstyleemphasis{sensor\_groups}

\item {} 
\sphinxstyleemphasis{subgroup\_phase\_center}

\item {} 
\sphinxstyleemphasis{linear\_coefficients}

\item {} 
\sphinxstyleemphasis{base\_numbers}

\item {} 
\sphinxstyleemphasis{k\_length}

\end{itemize}

\item {} 
results\_permutations: holds the results of the permutations and ambiguities.
\begin{itemize}
\item {} 
\sphinxstyleemphasis{intersections\_integers\_complete}

\item {} 
\sphinxstyleemphasis{ambiguity\_distances\_INT\_FORM\_MAT}

\item {} 
\sphinxstyleemphasis{ambiguity\_distances\_INT\_FORM\_mean}

\item {} 
\sphinxstyleemphasis{ambiguity\_distances\_WAVE\_FORM\_MAT}

\item {} 
\sphinxstyleemphasis{ambiguity\_distances\_WAVE\_FORM}

\item {} 
\sphinxstyleemphasis{intersection\_line}

\item {} 
\sphinxstyleemphasis{survivors}

\end{itemize}

\end{itemize}

\end{itemize}


\section{Use the results to see the ambiguities for a DOA.}
\label{\detokenize{test_example:use-the-results-to-see-the-ambiguities-for-a-doa}}
\begin{figure}[htbp]
\centering

\noindent\sphinxincludegraphics[scale=0.8]{{figure1}.png}
\end{figure}

\begin{figure}[htbp]
\centering

\noindent\sphinxincludegraphics[scale=0.8]{{figure2}.png}
\end{figure}

\begin{figure}[htbp]
\centering

\noindent\sphinxincludegraphics[scale=0.8]{{figure3}.png}
\end{figure}

\begin{figure}[htbp]
\centering

\noindent\sphinxincludegraphics[scale=0.8]{{figure4}.png}
\end{figure}

\begin{figure}[htbp]
\centering

\noindent\sphinxincludegraphics[scale=0.8]{{figure6}.png}
\end{figure}

\begin{figure}[htbp]
\centering

\noindent\sphinxincludegraphics[scale=0.8]{{figure7}.png}
\end{figure}


\chapter{Indices and tables}
\label{\detokenize{index:indices-and-tables}}\begin{itemize}
\item {} 
\DUrole{xref,std,std-ref}{genindex}

\item {} 
\DUrole{xref,std,std-ref}{modindex}

\item {} 
\DUrole{xref,std,std-ref}{search}

\end{itemize}


\renewcommand{\indexname}{Python Module Index}
\begin{sphinxtheindex}
\def\bigletter#1{{\Large\sffamily#1}\nopagebreak\vspace{1mm}}
\bigletter{a}
\item {\sphinxstyleindexentry{ambiguity\_calculator}}\sphinxstyleindexpageref{radar:\detokenize{module-ambiguity_calculator}}
\indexspace
\bigletter{f}
\item {\sphinxstyleindexentry{functions}}\sphinxstyleindexpageref{functions:\detokenize{module-functions}}
\indexspace
\bigletter{p}
\item {\sphinxstyleindexentry{plots\_generator}}\sphinxstyleindexpageref{plots_generator:\detokenize{module-plots_generator}}
\indexspace
\bigletter{r}
\item {\sphinxstyleindexentry{radarconf}}\sphinxstyleindexpageref{radarconf:\detokenize{module-radarconf}}
\end{sphinxtheindex}

\renewcommand{\indexname}{Index}
\printindex
\end{document}